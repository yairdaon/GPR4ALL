
\documentclass[paper=a4, fontsize=11pt]{scrartcl} % A4 paper and 11pt font size

\usepackage[T1]{fontenc} % Use 8-bit encoding that has 256 glyphs
\usepackage[english]{babel} % English language/hyphenation
\usepackage{amsmath,amsfonts,amsthm} % Math packages
\usepackage{algorithm} % algorithm package


\usepackage{sectsty} % Allows customizing section commands
\allsectionsfont{\centering \normalfont\scshape} % Make all sections centered, the default font and small caps

\usepackage{fancyhdr} % Custom headers and footers
\pagestyle{fancyplain} % Makes all pages in the document conform to the custom headers and footers
\fancyhead{} % No page header - if you want one, create it in the same way as the footers below
\fancyfoot[L]{} % Empty left footer
\fancyfoot[C]{} % Empty center footer
\fancyfoot[R]{\thepage} % Page numbering for right footer
\renewcommand{\headrulewidth}{0pt} % Remove header underlines
\renewcommand{\footrulewidth}{0pt} % Remove footer underlines
\setlength{\headheight}{13.6pt} % Customize the height of the header

%%%%%%%%%%%%%%%%%%%%%%%%%%%%%%%%%%%%%%%%%%%%%
\DeclareMathOperator*{\argmin}{arg\,min}
\newtheorem{theorem}{Theorem}[section]
\newtheorem{lemma}[theorem]{Lemma}
\newtheorem{proposition}[theorem]{Proposition}
\newtheorem{corollary}[theorem]{Corollary}
%%%%%%%%%%%%%%%%%%%%%%%%%%%%%%%%5
\numberwithin{equation}{section} % Number equations within sections (i.e. 1.1, 1.2, 2.1, 2.2 instead of 1, 2, 3, 4)
\numberwithin{figure}{section} % Number figures within sections (i.e. 1.1, 1.2, 2.1, 2.2 instead of 1, 2, 3, 4)
\numberwithin{table}{section} % Number tables within sections (i.e. 1.1, 1.2, 2.1, 2.2 instead of 1, 2, 3, 4)

\setlength\parindent{0pt} % Removes all indentation from paragraphs - comment this line for an assignment with lots of text





%----------------------------------------------------------------------------------------
% new commands
%----------------------------------------------------------------------------------------
\DeclareMathOperator*{\argmax}{arg\,max}

% boldface caligraphic etc
\newcommand{\ee}{\mathbb{E}}
\newcommand{\rr}{\mathbb{R}}
\newcommand{\logl}{\mathbf{l}}
\newcommand{\xx}{\mathbf{x}}
\newcommand{\cov}{\text{cov}}
\newcommand{\dd}{\mathcal{D}}
\newcommand{\egp}{\mathbb{E}_{\mathcal{GP}}}
\newcommand{\exo}{\mathbb{E}_{(\xn,\omega)}}
\newcommand{\var}{\text{Var}}
\newcommand{\vgp}{\text{Var}_{\mathcal{GP}}}
\newcommand{\bars}{ \ \ \ \ \ \ \ \ \ \ } 
% shortcuts
\newcommand{\ddxni}{\frac{\partial}{   \partial x_i^{(n)} }} % partial wrt ith coordinate of x^(n)
\newcommand{\gxn}{\nabla_{x^{(n)}}} % gradient wrt x^(n)
\newcommand{\xn}{x^{(n)}} 
\newcommand{\xii}{x^{(i)}}
\newcommand{\xj}{x^{(j)}}
\newcommand{\kn}{k^{(n)}}
\newcommand{\knm}{k^{(n-1)}}
\newcommand{\Kinvn}{K^{(-n)}} % K^(-n)
\newcommand{\Kinvnm}{K^{(-[n-1])}} 
\newcommand{\der}[2]{\frac{\partial #1}{\partial #2}}
\newcommand{\pr}{\text{pr}} % prior
\newcommand{\post}{\text{post}} % posterior
\newcommand{\norm}[1]{||#1||} %norm
\newcommand{\kinv}{K^{-1}} % K inverse
\newcommand{\sqn}{\sigma ^{2 , (n)}     }
\newcommand{\sqnm}{\sigma ^{2 , (n-1)}     }
\newcommand{\signxn}{\sigma^{2, (n-1)} (\xn)}
\newcommand{\munxn}{\mu ^{(n-1)}(\xn)}
\newcommand{\vns}{v^{(n)}(s)}
\newcommand{\vnssqr}{v^{2,(n)}(s)}
%\newcommand{\np}{^{(n)}} % mu^(n)
%\newcommand{\n1}{^{(n-1)}} % mu^(n-1)

%----------------------------------------------------------------------------------------
%	TITLE SECTION
%----------------------------------------------------------------------------------------


\begin{document}

\section{Notation}
\begin{itemize}
 \item $\rr^d$ the linear space over which we wish to integrate.
 \item Superscript in parenthesis is an index: $\xj \in \rr^d$ is the $j$th vector.
 \item Subscript is coordinate: $\xj_i$ is the $i$th coordinate of $\xj \in \rr^d$.
 \item $k: \rr ^d \times \rr ^d \to \rr$ the covariance function (kernel).
 \item $(\Omega, \mathcal{F}, \mathbb{P})$ the probability space of the gaussian process. Changes as we condition.
 \item $f : \rr^d \times \Omega \to \rr$ a gaussian process.
 \item $\kn: \rr^d \to \rr^n$ is such that $\kn_j(s) = k(\xj , s)$ where $1 \leq j \leq n$. 
 \item $K^{(n)} \in \rr^{ n\times n} $ is a Gram matrix: $K^{(n)}_{ij} = k(\xii , \xj)$.
 \item $\Kinvn = [K^{(n)}]^{-1}$.
 \item $\Kinvnm = [K^{(n-1)}]^{-1}$.
 \item Let $V: \rr^d \to \rr$ and $x \in \rr^d$. Then $\egp[ V(x) ]=\egp [V(x, \cdot)] =  \int_{\Omega} V(x, \xi) \mathbb{P}(d\xi) $
\end{itemize}


\section{Block Inversion}
We need to innvert covariance matrices. Moreover, we need to invert
$$ \left( \begin{array}{cc}
K   & k \\
k^t & c \end{array} \right)
$$

where $K$ is a symmetric matrix, $k$ is a vector and $c$ is a scalar.
The inverse is given by:

$$ \left( \begin{array}{cc}
\bar{K}   &  \bar{k} \\
\bar{k}^t   &  \bar{c} \end{array} \right)
$$
with:
\begin{align}
 \begin{split}
\bar{K} &=  \kinv + K^{-1}k\bar{c}k^t\kinv\\
\bar{k} &=  -\kinv k\bar{c}\\
\bar{c} &= (c - k^t\kinv k)^{-1} 
 \end{split}
\end{align}

\section{A Lemma}
We use the block inversion to prove the following:
\begin{lemma}
Let $f_n = \mu^{(n-1)}(\xn) =  k^{(n-1)}(\xn)K^{(-[n-1])}f^{(n-1)}$. Then $\mu^{(n-1)}(s) = \mu^{(n)}(s)$.
\end{lemma}
\begin{proof}
 For this proof we fix $s\in \rr ^{d}$ and denote:
\begin{itemize}
 \item $\kinv := K^{(-[n-1])}$ 
 \item $k : = k^{(n-1)}(\xn)$
 \item $c := k(\xn,\xn)$
 \item $f = f^{(n-1)}$ 
 \item $a : = \mu^{(n-1)}(\xn) = k^t\kinv f$ (using the above notation)
 \item $k(s) = k^{(n-1)}(s)$
 \item $\bar{c} := (c - k^t\kinv k)^{-1}$. 
 \item $a = \mu(s) : = \mu^{(n-1)}(s)$
\end{itemize}
Then $\mu^{(n-1)}(s) = k^t\kinv f$ and we may express $\mu^{(n)}(s)$ using the block inversion.


\begin{align}
 \begin{split}
%
\mu^{(n)}(s) &= \left( \begin{array}{cc}
 k^t(s)  & k(\xn, s)  
\end{array} \right)
%
%
\left( \begin{array}{cc}
\kinv + \kinv k \bar{c} k^t \kinv  &  -\kinv k \bar{c} \\
-k^t\kinv \bar{c}   &  \bar{c}
 \end{array} \right)
%
%
\left( \begin{array}{cc}
 f  \\ a 
 \end{array} \right) \\\\
%
%
%
%
% new line
%
%
%
%
&= \left( \begin{array}{cc}
 k^t(s)  & k(\xn, s)
 \end{array} \right)
%
%
\left( \begin{array}{cc}
\kinv + \kinv k \bar{c} k^t \kinv  &  -\bar{c}\kinv k  \\
-\bar{c}k^t\kinv   & \bar{c}
 \end{array} \right)
%
%
\left( \begin{array}{cc}
 f  \\ a
  \end{array} \right) \\\\
%
%
&=  k^t(s)(\kinv + \kinv k \bar{c}k^t \kinv )f - a\bar{c} k^t(s) \kinv k  -\bar{c} k(\xn,s) k^t\kinv f + \bar{c}a k(\xn,s) \\\\
%
%%
%
&=  k^t(s)\kinv f + \bar{c}[k^t(s)\kinv k\cdot  k^t \kinv f - a k^t(s) \kinv k \\
 &\ \ \ \ \ \ \ \ \ \ \ \ \ - k(\xn, s) k^t\kinv f + a k(\xn,s)] \\\\
%
%
%
&= \mu (s) + \bar{c}[k^t(s)\kinv k\cdot  \mu (\xn) -\mu(\xn) k^t(s) \kinv k \\
 &\ \ \ \ \ \ \ \ \ \ \ \ \ - k(\xn, s) \mu(\xn) + \mu(\xn) k(\xn,s)] \\\\
%
%
%
& = \mu(s) = \mu^{(n-1)} (s)
\end{split}
\end{align}
\end{proof}

The bottom line is that the observation $f( \xn ,\omega) = \mu^{(n-1)}(\xn)$ changes nothing.




\section{Average Variance}



We seek $\argmin_{\xn}$ of
\begin{align}
 \begin{split}
%
&\int  \vgp[ \exp( k^{(n)}(s)K^{(-n)}f + f_\pr(s) )]  ds  \\\\
%
&=\int e^{2f_\pr(s)}  \vgp[ \exp(\sum_{i,j=1}^{n} k(s,\xii)K_{ij}^{(-n)}f_j )  ]  ds\\\\
%
&=\int e^{2f_\pr(s)}  \vgp[ \exp(\sum_{i=1}^{n}\sum_{j=1}^{n-1}  k(s,\xii)K_{ij}^{(-n)}f_j + \sum_{i=1}^n k(s,\xii)K_{in}^{(-n)}f_n(\omega) )  ]  ds\\\\
%
&=\int e^{2f_\pr(s)}  \exp(2\sum_{i=1}^{n}\sum_{j=1}^{n-1}  k(s,\xii)K_{ij}^{(-n)}f_j)  \vgp[\exp( \sum_{i=1}^n k(s,\xii)K_{in}^{(-n)}f_n(\omega) ) ] ds\\\\
%
&=\int \exp(2\sum_{i=1}^{n}\sum_{j=1}^{n-1}  k(s,\xii)K_{ij}^{(-n)}f_j + 2f_\pr(s)) \vgp [\exp( \vns f_n )]  ds,\\\\
%
 \end{split}
\end{align}


where we denote $\vns := \sum_{i=1}^n k(s,\xii)\Kinvn_{in}$. By our assumptions, $\vns f_n(\omega) \sim \mathcal{N}(\vns\munxn,
\vnssqr \signxn)$.
This means that 

$$
\vgp[e^{\vns f_n(\omega)}] = (e^{\vnssqr}\signxn - 1) \exp( 2\vns\munxn + \vnssqr\signxn )
$$

We plug this in the above series of equalities and define $f_n :=\munxn, \sigma^2: = \signxn$:

\begin{align}
 \begin{split}
&= \int  \exp(2\sum_{i=1}^{n}\sum_{j=1}^{n-1}  k(s,\xii)K_{ij}^{(-n)}f_j + 2f_\pr(s)) \cdot 
	      (e^{\vnssqr\sigma^2 } - 1)\exp( 2\vns f_n + \vnssqr \sigma^2 )  ds,\\
%
&= \int \exp(2\sum_{i=1}^{n}\sum_{j=1}^{n-1}  k(s,\xii)K_{ij}^{(-n)}f_j + 2\vns f_n + 2f_\pr(s)) 
	(e^{\vnssqr \sigma^2 } - 1)\exp( \vnssqr \sigma^2 )  ds,\\
%%
&= \int \exp(2\kn(s)\Kinvn f + 2f_\pr(s)) \cdot  (e^{\vnssqr \sigma^2 } - 1)\exp( \vnssqr \sigma^2 )  ds\\\\
%
%
%
&= \int \exp(2\kn(s)\Kinvn f + 2f_\pr(s)) (e^{2 \vnssqr\sigma^2 } - e^{\vnssqr \sigma^2 }) ds\\\\
%
%
%
&=\int \exp(2k^{(n-1)}(s)\Kinvnm f + 2f_\pr(s)) (e^{2\vnssqr \sigma^2 } - e^{\vnssqr \sigma^2 }) ds\\\\ 
%
%
%
%
&=\int \exp(2\mu^{(n-1)}(s)) (e^{2\vnssqr \sigma^2 } - e^{\vnssqr \sigma^2 }) ds \text{ (by the lemma) }\\\\ 
%
%
\end{split}
\end{align}

We now focus on the exponents using the block inversion. We take $k = \kn(\xn), K = \Kinvnm, c = k(\xn,\xn)$ and note 
that in the matrix block inversion, $\bar{c} =\frac{1}{\sqnm (\xn)}$

\begin{align}
 \begin{split}
%
%
  \vnssqr \sqnm(\xn) &= [\sum_{i=1}^n k(s,\xii)\Kinvn_{in}]^2\sqnm(\xn)  \\\\
% 
%
%
  & [k(s, \xn) \bar{c} + < k^{(n-1)}(s) ,  -\bar{c}\Kinvnm k^{(n-1)}(\xn) >]^2\sqnm(\xn)  \\\\
% 
%
%
  &= \frac{[k(s, \xn) - < k^{(n-1)}(s) ,  \Kinvnm k^{(n-1)}(\xn) > ]^2}{\sqnm(\xn)} \\
\end{split}
\end{align}

We need to differentiate this wrt $\xn$.



\section{Derivatives}
In what follows, we temporarily denote $u(\xn) = k(s, \xn)  - < \knm (s) ,  \Kinvnm \knm (\xn) > $, suppressing 
the dependence on $s$. 
\begin{align}
 \begin{split}
  \gxn \left[ \vnssqr\sqnm(\xn) \right] &= \gxn \frac{[k(s, \xn)  - < \knm (s) ,  \Kinvnm \knm (\xn) > ]^2}{\signxn}\\\\
%
%
%
%
&= \gxn \frac{ u^2 (\xn) }{\signxn}\\\\
%
%
%
&=\frac{ 2u(\xn)\gxn u(\xn) \signxn}{(\signxn)^2} \\
	    &\bars- \frac{\gxn \signxn u^2(\xn)}{(\signxn)^2} \\\\
%
%
%
&=\frac{ 2u(\xn)}{\signxn} \gxn u(\xn)\\
	    &\bars- \frac{u^2(\xn)}{(\signxn)^2}\gxn \signxn \\\\
%
%
%
%
&=\frac{ 2u(\xn)}{\signxn} \gxn k(s , \xn) \\
	    &\bars - \frac{ 2u(\xn)}{\signxn} \gxn< \knm (s) ,  \Kinvnm \knm (\xn) > \\
	    &\bars - \frac{u^2(\xn)}{(\signxn)^2}\gxn \signxn \\\\
%
%
%
%
%
%
%
 \end{split}
\end{align}


In what follows, we hold $n$ fixed and denote $g(s) : = \vnssqr \sqnm(\xn)$ and $\nabla g(s)$ its gradient wrt $\xn$. We wish to find 
$\argmin_{\xn}$ of the average variance and so we need to find derivatives.

\begin{align}
 \begin{split}
 & \gxn \int \exp(2\mu^{(n-1)}(s)) (e^{2\vnssqr \sigma^2 } - e^{\vnssqr\sigma^2 }) ds \\
%
%
% 
&=  \int \exp(2\mu^{(n-1)}(s)) \gxn (e^{2\vnssqr\sigma^2 } - e^{\vnssqr\sigma^2 }) ds \\
%
%
%
%
%
% 
&=  \int \exp(2\mu^{(n-1)}(s)) \nabla (e^{2g(s) } - e^{g(s)}) ds \\
%
%
% 
%
%
%
% 
&=  \int \exp(2\mu^{(n-1)}(s)) (2e^{2g(s) } - e^{g(s)})\nabla g(s)\  ds \\
%
%
% 
%
%
%
% 
&\approx \frac{1}{n_s}  \sum_{k=1}^{n_s} (2e^{2g(Z_k) } - e^{g(Z_k)})\nabla g(Z_k), \\
%
%
% 
 \end{split}
\end{align}

where $Z_k$ are drawn using the log likelihood $2\mu^{(n-1)}(\cdot)$ (note the $2$ in there). We do not need to normalize it since it does not depend 
on $x^{(n)}$ and so this normalization is only a constant factor.


\section{Xtras}
Start with easy sneezy.
\begin{align}
 \begin{split}
  \gxn k(\xn,s) &= \gxn c\exp(-\frac{||\xn - s||^2}{2r^2} )\\
%
&= -k(\xn,s) \gxn\frac{||\xn - s||^2}{2r^2} \\
%
&= -\frac{k(\xn,s)}{2r^2} \gxn||\xn - s||^2 \\
%
&= -\frac{k(\xn ,s)}{r^2} (\xn-s)\\
 \end{split}
\end{align}

In the following, gradient is taken with respect to $s$.

\begin{align}
 \begin{split}
  \nabla \sqn (s) &= \nabla [ c - \kn(s) \Kinvn \kn(s)]\\\\
%
%
%
&= -2 \kn(s) \Kinvn \nabla\kn(s)\\\\
%
%
%
&= -2 \sum_{i,j=1}^{n} \kn(s)_j \Kinvn_{ji} \nabla k (s, \xii )\\\\
%
%
%
&= 2 \sum_{i,j=1}^{n} k(s,\xj) \Kinvn_{ji} \frac{k(\xii ,s)}{r^2} (s-\xii)\\\\
%
%
%
&= \frac{2}{r^2} \sum_{i,j=1}^{n} k(s,\xj) \Kinvn_{ji} k(\xii ,s) (s-\xii)\\\\
 \end{split}
\end{align}

Consequently:

\begin{align}
 \begin{split}
  \gxn \signxn &= \frac{2}{r^2} \sum_{i,j=1}^{n-1} k(\xn,\xj) \Kinvnm_{ji} k(\xii ,\xn) (\xn-\xii)\\\\
 \end{split}
\end{align}


\end{document}

